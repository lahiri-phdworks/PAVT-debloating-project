\documentclass{article} % For LaTeX2e
\usepackage{nips_adapted}
\usepackage{hyperref}
\usepackage{url}
\usepackage{amsmath}
\usepackage{graphicx}
\usepackage{amssymb}
\usepackage{hyperref}
\hypersetup{
    colorlinks=true,
    linkcolor=blue,
    filecolor=magenta,      
    urlcolor=blue,
}

\title{CS 639A Progress Report 1}

\author{
Sumit Lahiri \\
19111274 \\
\And
Group No\\
1\\
\And
Amit Kumar Sharma \\
20111012\\
}

% The \author macro works with any number of authors. There are two commands
% used to separate the names and addresses of multiple authors: \And and \AND.
%
% Using \And between authors leaves it to \LaTeX{} to determine where to break
% the lines. Using \AND forces a linebreak at that point. So, if \LaTeX{}
% puts 3 of 4 authors names on the first line, and the last on the second
% line, try using \AND instead of \And before the third author name.

\newcommand{\fix}{\marginpar{FIX}}
\newcommand{\new}{\marginpar{NEW}}
\usepackage{xcolor}

\nipsfinalcopy


\begin{document}

\maketitle

\section*{First \& Second Sprint : Brief Progress}
We started investigating the problem in $\textbf{three splits}$ for our $\textbf{Pre-RL Stage Work}$. The most \textbf{time-consuming} part was to migrate the outdated packages used in some the tools chosen by us for delivering the project. We gained a fair idea on how $\textbf{embedding spaces}$ are use and the novelty in $\textbf{inst2vec}$ tool. We started running the $\textbf{inst2vec}$ tool to generate an embedding for our debloating sample space, like $\textbf{call-site}$ information, $\textbf{function arguments}$, $\textbf{type parameters}$ etc as described in \href{http://www.csl.sri.com/users/gehani/papers/MLSys-2019.DeepOCCAM.pdf}{Paper 2}. 

\section*{Stage 1 : Split 1} 
\begin{itemize}
    \item Completed the setup for LLVM-IR generation, clang tool, \textbf{spcl/ncc (inst2vec)} tool. 
    \item Completed the setup for $\textbf{OCCAM}$ tool, $\textbf{Chisel}$ tool \& $\textbf{Trimmer}$ tool 
	\item We are in the middle of running the \textbf{inst2vec} tool and using the \textbf{public data-sets} available to train it. 
\end{itemize}

\section*{Stage 1 : Split 2}
\begin{itemize}
	\item We found \textbf{2} more features where \textbf{program debloating} may be possible. 
	\item We understood how to write specification for the script used by \textbf{Chisel} tool as in \href{https://dl.acm.org/doi/10.1145/3243734.3243838}{Paper 1}
\end{itemize}

\section*{Stage 1 : Split 3}
\begin{itemize}
	\item Bloated Sample collection and preparation started. 
\end{itemize}

\end{document}
