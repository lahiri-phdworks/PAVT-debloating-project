\documentclass{article} % For LaTeX2e
\usepackage{nips_adapted}
\usepackage{hyperref}
\usepackage{url}
\usepackage{amsmath}
\usepackage{graphicx}
\usepackage{amssymb}
\usepackage{hyperref}
\hypersetup{
    colorlinks=true,
    linkcolor=blue,
    filecolor=magenta,      
    urlcolor=blue,
}

\title{CS 639A Progress Report 1}

\author{
Sumit Lahiri \\
19111274 \\
\And
Group No\\
1\\
\And
Amit Kumar Sharma \\
20111012\\
}

% The \author macro works with any number of authors. There are two commands
% used to separate the names and addresses of multiple authors: \And and \AND.
%
% Using \And between authors leaves it to \LaTeX{} to determine where to break
% the lines. Using \AND forces a linebreak at that point. So, if \LaTeX{}
% puts 3 of 4 authors names on the first line, and the last on the second
% line, try using \AND instead of \And before the third author name.

\newcommand{\fix}{\marginpar{FIX}}
\newcommand{\new}{\marginpar{NEW}}
\usepackage{xcolor}

\nipsfinalcopy


\begin{document}

\maketitle

\section*{First \& Second Sprint : Brief Progress Report}
We started working on the \textbf{\href{https://github.com/lahiri-phdworks/reinforcedlearning-debloater}{debloating project}} in $\textbf{three splits}$ for our $\textbf{Pre-RL Stage Work}$. The most \textbf{time-consuming} part was to migrate the outdated packages used in some the tools chosen by us for delivering the project. We gained a fair idea on how $\textbf{embedding spaces}$ are used and the novelty in \textbf{\href{https://github.com/lahiri-phdworks/ncc}{inst2vec}} tool as modelling an LLVM IR as \textbf{XFG} \& mapping to an \textbf{NLP} like \textbf{skip-gram} model. We started running the \textbf{\href{https://github.com/lahiri-phdworks/ncc}{inst2vec}} tool to generate an embedding for our debloating \texttt{C++} sample space, like $\textbf{call-site}$ information, $\textbf{function arguments}$, $\textbf{type parameters}$ etc as described in \textbf{\href{http://www.csl.sri.com/users/gehani/papers/MLSys-2019.DeepOCCAM.pdf}{Paper 2}}. We are reporting the highlights from sprint of \textbf{\href{https://github.com/lahiri-phdworks/reinforcedlearning-debloater/commit/dd63911895cb31cdc77c9debd57090422ffb3b65}{Commit-1 01/11/2020}} to 
\textbf{\href{https://github.com/lahiri-phdworks/reinforcedlearning-debloater/commit/8066e7b7b7fa7513e3611e3f3aee6fb72c81ea83}{Commit-22 09/11/2020}} which is \textbf{Week-1 : November 2020}.

\section*{Proposed Stage 1 Milestones : Pre RL Stage :: Week 1 \& 2}
\begin{itemize}
	\item Start with $\textbf{LLVM IR}$ of the code and use $\textbf{Inst2Vec}$ as outlined in \href{http://www.csl.sri.com/users/gehani/papers/MLSys-2019.DeepOCCAM.pdf}{Paper 2} to extract feature information as outlined in the paper before $\textbf{RL stage}$
	\item Finding more sites other than the ones listed for $\textbf{Program Specialization}$ in \href{http://www.csl.sri.com/users/gehani/papers/MLSys-2019.DeepOCCAM.pdf}{Paper 2}.  
	\item Understanding $\textbf{Chisel}$ \& $\textbf{Trimmer}$ tool as implemented in \href{https://dl.acm.org/doi/10.1145/3243734.3243838}{Paper 1} in terms of how \textbf{markov decision process} enhances \textbf{delta debugging}. This steps helps in \textbf{Stage 2}.
	\item Collect and prepare bloated code samples for analysis, this is the input to our tool infrastructure. We will explore the use of fuzzing here.  
\end{itemize}

\textbf{\color{teal} We list below the progress on the above points that we have been able to achieve so far.}

\section*{Stage 1 : Split 1} 
\begin{itemize}
    \item Completed the setup for LLVM-IR generation, clang tool, \textbf{\href{https://github.com/lahiri-phdworks/ncc}{inst2vec}} tool. 
    \item Completed the setup for \textbf{\href{https://github.com/ashish-gehani/OCCAM}{OCCAM}} tool, \textbf{\href{http://www.csl.sri.com/users/gehani/papers/ASE-2018.Trimmer.pdf}{Trimmer}} tool \& \textbf{\href{https://github.com/aspire-project/chisel}{Chisel}} tool. All setup \& build related issues were \textbf{\color{teal}resolved} and \textbf{\color{teal}ran} once one sample example.
	\item We are in the middle of running the \textbf{\href{https://github.com/lahiri-phdworks/ncc}{inst2vec}} tool and using the \textbf{public data-sets} available to train it. 
	\item We are using the pre-trained embeddings from \textbf{\href{https://github.com/lahiri-phdworks/ncc}{inst2vec}} repository for the starting point for the \textbf{DeepOCCAM} tool we are developing.
\end{itemize}

\section*{Stage 1 : Split 2}
\begin{itemize}
	\item We found \textbf{2} more features where \textbf{program debloating} may be possible. 
	\item We understood how to write specification for the script used by \textbf{Chisel} tool as in \href{https://dl.acm.org/doi/10.1145/3243734.3243838}{Paper 1} in terms of the three parts of the scripts namely \textbf{\texttt{compile()}}, \textbf{\texttt{desired()}} \& \texttt{\textbf{undesired()}}.
	\item \textbf{Chisel} tool works by learning a \textbf{policy} for \textbf{delta debugging} by \textbf{reinforcement learning} which guarantees \textbf{1-minimal P*} \& $\textbf{$O(|P|^2)$}$ runtime. The abstraction is that a \textbf{markov decision process} is being used to model the \textbf{reinforcement learning} problem for meaningful \textbf{guidance} to learn the \textbf{policy} in a better way.
	\item We spend time understanding how \textbf{\href{https://github.com/ashish-gehani/OCCAM}{OCCAM}} tool, \textbf{\href{http://www.csl.sri.com/users/gehani/papers/ASE-2018.Trimmer.pdf}{Trimmer}} tool \& \textbf{\href{https://github.com/aspire-project/chisel}{Chisel}} tool are working.
\end{itemize}

\section*{Stage 1 : Split 3}
\begin{itemize}
	\item Bloated Sample collection and preparation started, We are targetting \href{https://github.com/bitcoin/bitcoin}{bitcoin1-src} \& \href{https://github.com/ethereum/solidity}{eth-solidity}. Past experience tells that the code is most probably \textbf{bloated}. We may have to change it later to publicly available samples. \textbf{\color{red}[This needs some more work]}.
	\item We found this \textbf{\href{https://www.youtube.com/watch?v=hC4zIwyv1bg}{video}} useful to understand how \textbf{program debloating} can be improved using \textbf{Stochastic Optimization}, which we intend to explore as \textbf{extra work} if time permits.
\end{itemize}

We are almost \textbf{\color{teal}ready} to move to the next \textbf{Stage 2} of modelling and developing the \textbf{\color{teal} reinforcement learning model} using \textbf{PyTorch}. We intend to wrap up an left over tasks from \textbf{Stage 1} and move on to working on \textbf{Stage 2} in the next couple of days.  

\end{document}
